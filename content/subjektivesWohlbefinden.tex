%%%%%%%%%%%%%%%%%%%%%%%%%%%%%%%%%%%%%%%%%%%%%%%%%%%%%%%%%%%%%%%%%
%_____________ ___    _____  __      __ 
%\____    /   |   \  /  _  \/  \    /  \  Institute of Applied
%  /     /    ~    \/  /_\  \   \/\/   /  Psychology
% /     /\    Y    /    |    \        /   Zürcher Hochschule 
%/_______ \___|_  /\____|__  /\__/\  /    fuer Angewandte Wissen.
%        \/     \/         \/      \/                           
%%%%%%%%%%%%%%%%%%%%%%%%%%%%%%%%%%%%%%%%%%%%%%%%%%%%%%%%%%%%%%%%%
%
% Project     : Seminararbeit
% Title       : 
% File        : subjektivesWohlbefinden.tex Rev. 00
% Date        : 10.10.2012
% Author      : Till J. Ernst
%
%%%%%%%%%%%%%%%%%%%%%%%%%%%%%%%%%%%%%%%%%%%%%%%%%%%%%%%%%%%%%%%%%
\thispagestyle{empty}
\chapter{Subjektives Wohlbefinden}\label{chap.swb}
%Def. Acronym
\newacronym{swb}{SWB}{Subjektives Wohlbefinden}
\newacronym{sm}{SM}{Soziale Medien}
%HK-Subjektives Wohlbefinden
Dieses Kapitel befasst sich mit der Begriffserklärung und der Abgrenzung des Begriffs \textit{\gls{swb}}. Des Weiteren wird auf die historische Entstehung, die Verwendung in der positiven Psychologie und auf das psychologische Konstrukt des \gls{swb} eingegangen.
%UK-Begriffserklärung
\section{Begriffserklärung}\label{begriff}
Der Begriff \textit{Wohlbefinden} wird in der Literatur sehr unterschiedlich beschrieben. Eine eindeutige Erklärung ist somit kaum möglich. Begriffe wie z.B. Glück, Glücksempfinden, Lebenszufriedenheit, Lebensqualität, positive Emotionen, positive Affekte u.v.a., werden in der Literatur teilweise identisch behandelt, teilweise aber auch voneinander abgegrenzt. Im Folgenden sollen einige Begriffe näher erläutert werden, um die unterschiedliche Verwendung ein wenig näher zu beleuchten.\newline
Gemäss Seligman  \cite{Seligman:2003} sind die Worte \textit{Glück} oder \textit{Glücklichkeit} und \textit{Wohlbefinden} austauschbar und umfassen positive Gefühle, wie Ekstase oder Behagen und positive Verhaltensweisen, die nicht über eine Gefühlskomponente verfügen, wie Absorption oder Engagement. Demzufolge lässt sich daraus schliessen, dass Glück und Wohlbefinden sich manchmal auf Gefühle, manchmal auf Aktivitäten, in denen nichts gefühlt wird, beziehen.\newline
Eine weitere Ref nach \cite{ackema:1998}


%UK-Begriffsabgrenzung
\section{Begriffsabgrenzung}\label{abgrenzung}

%UK-Historischer Überblick
\section{Historischer Überblick}\label{überblick}

%UK-SWB in der positiven Psychologie
\section{Verwendung in der positiven Psychologie}\label{positivePsychologie}

%UK-Psychologisches Konzept
\section{Psychologisches Konzept}\label{konzept}


