%%%%%%%%%%%%%%%%%%%%%%%%%%%%%%%%%%%%%%%%%%%%%%%%%%%%%%%%%%%%%%%%%
%_____________ ___    _____  __      __ 
%\____    /   |   \  /  _  \/  \    /  \  Institute of Applied
%  /     /    ~    \/  /_\  \   \/\/   /  Psychology
% /     /\    Y    /    |    \        /   Zürcher Hochschule 
%/_______ \___|_  /\____|__  /\__/\  /    fuer Angewandte Wissen.
%        \/     \/         \/      \/                           
%%%%%%%%%%%%%%%%%%%%%%%%%%%%%%%%%%%%%%%%%%%%%%%%%%%%%%%%%%%%%%%%%
%
% Project     : Seminararbeit
% Title       : 
% File        : subjektivesWohlbefinden.tex Rev. 00
% Date        : 10.10.2012
% Author      : Till J. Ernst
%
%%%%%%%%%%%%%%%%%%%%%%%%%%%%%%%%%%%%%%%%%%%%%%%%%%%%%%%%%%%%%%%%%
\thispagestyle{empty}
\chapter{Subjektives Wohlbefinden}\label{chap.swb}
%Def. Acronym
\newacronym{swb}{SWB}{Subjektives Wohlbefinden}
\newacronym{sm}{SM}{Soziale Medien}
\newacronym{aw}{AW}{aktuelles Wohlbefinden}
\newacronym{hw}{HW}{habituelles Wohlbefinden}
%HK-Subjektives Wohlbefinden
TBD\newline
Dieses Kapitel befasst sich mit der Begriffserklärung und der Abgrenzung des Begriffs \textit{\gls{swb}}. Des Weiteren wird auf die historische Entstehung, die Verwendung in der positiven Psychologie und auf das psychologische Konstrukt des \gls{swb} eingegangen.
%UK-Begriffserklärung
\section{Begriffserklärung}\label{begriff}
\glsreset{swb}
Der Begriff \textit{Wohlbefinden} und mit ihm der zusammengesetzte Begriff \textit{Subjektives Wohlbefinden} wird in der Literatur sehr unterschiedlich beschrieben. Eine eindeutige Erklärung ist somit kaum möglich. Begriffe wie z.B. Glück, Glücksempfinden, Lebenszufriedenheit, Lebensqualität, positive Emotionen, positive Affekte u.v.a., werden teilweise identisch behandelt, teilweise aber auch voneinander abgegrenzt. Um den Begriff \textit{\gls{swb}} zu beschreiben und zu erklären, sind unterschiedliche Vorgehensweisen denkbar. Es könnte die Herkunft und die historische Entwicklung des Begriffs in der Sprache zugezogen werden (\textit{etymologische} Erklärung). Den heutigen Wortgebrauch, wie er in den Lexika festgelegt ist (\textit{lexikalische} Erklärung) oder man könnte Menschen dazu befragen, was sie unter dem Begriff verstehen (\textit{empirische} Erklärung). 
Im Folgenden werden einige Begriffe anhand bisheriger Konzeptionen und Theorieansätzen erläutert, um Transparenz in die unterschiedliche Verwendung zu bringen und um ein für diese Arbeit notwendiges Wissen zu vermitteln, wie das \gls{swb} ind den folgenden Kapiteln zu verstehen ist.\newline
Gemäss Seligman  \fullcite{Seligman:2003} sind die Worte \textit{Glück} oder \textit{Glücklichkeit} und \textit{Wohlbefinden} austauschbar und umfassen positive Gefühle, wie Ekstase oder Behagen und positive Verhaltensweisen, die nicht über eine Gefühlskomponente verfügen, wie Absorption oder Engagement. Demzufolge lässt sich daraus schliessen, dass Glück und Wohlbefinden sich manchmal auf Gefühle, manchmal auf Aktivitäten, in denen nichts gefühlt wird, beziehen.\newline
Mayring \cite{Mayring:1991} versteht unter Glück ein Erleben auf einer emotionaler, kognitiver und aktionaler Ebene, das die ganze Persönlichkeit betrifft, das sich das ganze Leben lang ausbaut und sich über ein Hinausgehen über die eigene Ich-Bezogenheit erstreckt. \textquotedblleft Glück dient zur Vermittlung von Harmonie mit sich selbst und der Welt, Werthaftigkeit des Lebens und Gewinn neuer Gefühle und Erfahrungen, es kann vor Depressionen schützen.\textquotedblright \quad \cite[S.179]{Mayring:1991} \newline
Veenhoven \cite{Veenhoven:1991} verwendet das Wort Glück (\textit{happiness}) gleichbedeutend für Lebenszufriedenheit (\textit{life satisfaction}) und beschreibt das Ausmass an Qualität, was das Leben als Ganzes für ein Individuum bereit stellt. Vereinfacht gesagt, wie sehr ein Individuum das von ihm geführte Leben mag. Veenhoven unterscheidet zwischen einem affektiven und einem kognitiven Aspekt, was das Leben als Ganzes betrifft. Der affektive Aspekt betrifft das Niveau der Lust, resp. wie gut sich eine Person im Allgemeinen fühlt  und der kognitive Aspekt steht für den Grad der Zufriedenheit mit dem, was die Person denkt im Leben erreicht zu haben.\newline
Becker \cite{Becker:1991} unterscheidet zwischen \textit{\gls{aw}}, das die aktuelle Befindlichkeit eines Menschen charakterisiert und \textit{\gls{hw}}, welches als relativ stabile Eigenschaft eines Menschen zu verstehen ist. Becker verwendet \gls{aw} als Oberbegriff für die Beschreibung des momentanen Erlebens einer Person. Dieses beinhaltet positive getönte Gefühle, Stimmungen und körperliche Empfindungen sowie das Fehlen von Beschwerden. Im Gegensatz zum \gls{aw} werden mit dem \gls{hw} \textit{Urteile} über das für eine Person typische Wohlbefinden charakterisiert. Unter Urteile sind die Aussagen einer Person über \gls{hw} gemeint, die primär durch kognitive Prozesse zustande kommen.

%UK-Begriffsabgrenzung
\section{Begriffsabgrenzung}\label{abgrenzung}

%UK-Historischer Überblick
\section{Historischer Überblick und Entstehun}\label{überblick}
An dieser Stelle soll ein Ausflug in die Philosophie vorgenommen werden, da Glück von Anbeginn der philosphisch behandelt wurde und als ein zentraler Begriff in vieler philosophischer Systemen gilt \cite{Mayring:1991}. Eine Beschränkung erfolgt auf die abendländische Philosophie. Einerseits weil die abendländische Philosophie für unser Denken bestimmend gewesen ist und andererseits weil eine ausführliche Beschäftigung mit den restlichen Denkweisen, wie chinesische oder indische Philosophie, den Rahmen dieser Arbeit sprengen würde. \newline
In der vorphilosophischen Epoche des alten Griechenlands galt Glück gemäss Mayring \cite{Mayring:1991} als eine Gabe der Götter. Diese Gabe wurde als Reichtum, Ehre, Macht und Gesundheit symbolisiert. Erst der vorsokratiker Demokrit (ca. 470 bis ca. 370 v. Chr.) beschäftigte sich mit dem Thema, was der Mensch tun könne, um glücklich zu sein. Demokrit verwies darauf, dass Glück \textquotedblleft darüber hinaus und in stärkerem Mass von der inneren Verfassung des Menschen abhängt \textquotedblright \cite{Becker:1994} und nicht mehr durch äussere Güter, sondern durch die innere Haltung begründet ist \cite{Mayring:1991}. Diese Lehre wurde bestimmend für die griechische Lehre der Glückseligkeit und wurde in den Mittelpunkt der platonischen Philospohie des tugendhaften guten Lebens gestellt (ebda., 1991).\newline
 

%UK-SWB in der positiven Psychologie
\section{Verwendung in der positiven Psychologie}\label{positivePsychologie}

%UK-Psychologisches Konzept
\section{Psychologisches Konzept}\label{konzept}


