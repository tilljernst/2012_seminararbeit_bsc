%%%%%%%%%%%%%%%%%%%%%%%%%%%%%%%%%%%%%%%%%%%%%%%%%%%%%%%%%%%%%%%%%
%_____________ ___    _____  __      __ 
%\____    /   |   \  /  _  \/  \    /  \  Institute of Applied
%  /     /    ~    \/  /_\  \   \/\/   /  Psychology
% /     /\    Y    /    |    \        /   Zürcher Hochschule 
%/_______ \___|_  /\____|__  /\__/\  /    fuer Angewandte Wissen.
%        \/     \/         \/      \/                           
%%%%%%%%%%%%%%%%%%%%%%%%%%%%%%%%%%%%%%%%%%%%%%%%%%%%%%%%%%%%%%%%%
%
% Project     : Seminararbeit
% Title       : 
% File        : subjektivesWohlbefinden.tex Rev. 00
% Date        : 10.10.2012
% Author      : Till J. Ernst
%
%%%%%%%%%%%%%%%%%%%%%%%%%%%%%%%%%%%%%%%%%%%%%%%%%%%%%%%%%%%%%%%%%
\chapter{Subjektives Wohlbefinden}\label{chap.swb}
\glsresetall
Dieses Kapitel befasst sich mit der Begriffserklärung und der Abgrenzung des Begriffs \textquotedblleft subjektives Wohlbefinden\textquotedblright. Des Weiteren wird auf die historische Entstehung, die Verwendung in der positiven Psychologie und auf das psychologische Konstrukt des subjektiven Wohlbefindens eingegangen.
%UK-Begriffserklärung
\section{Begriffserklärung}\label{begriff}
Der Begriff Wohlbefinden und mit ihm der zusammengesetzte Begriff \gls{swb} wird in der Literatur sehr unterschiedlich beschrieben. Eine eindeutige Erklärung ist somit kaum möglich. Begriffe wie z.B. Glück, Glücksempfinden, Lebenszufriedenheit, Lebensqualität, positive Emotionen, positive Affekte u.v.a., werden teilweise identisch behandelt, teilweise aber auch voneinander abgegrenzt. Um den Begriff \gls{swb} zu beschreiben und zu erklären sind unterschiedliche Vorgehensweisen denkbar. Es könnte die Herkunft und die historische Entwicklung des Begriffs in der Sprache zugezogen werden (etymologische Erklärung). Den heutigen Wortgebrauch, wie er in den Lexika festgelegt ist (lexikalische Erklärung) oder man könnte Menschen dazu befragen, was sie unter dem Begriff verstehen (empirische Erklärung). 
Im Folgenden werden einige Begriffe anhand bisheriger Konzeptionen und Theorieansätzen erläutert, um Transparenz in die unterschiedliche Verwendung zu bringen und um ein für diese Arbeit notwendiges Wissen zu vermitteln, wie das \gls{swb} in den folgenden Kapiteln zu verstehen ist.\newline
Gemäss Seligman  \cite{Seligman:2003} sind die Worte Glück oder Glücklichkeit und Wohlbefinden austauschbar und umfassen positive Gefühle, wie Ekstase oder Behagen und positive Verhaltensweisen, die nicht über eine Gefühlskomponente verfügen, wie Absorption oder Engagement. Demzufolge lässt sich daraus schliessen, dass Glück und Wohlbefinden sich manchmal auf Gefühle, manchmal auf Aktivitäten, in denen nichts gefühlt wird, beziehen.\newline
Mayring \cite{Mayring:1991} versteht unter Glück ein Erleben auf einer emotionaler, kognitiver und aktionaler Ebene, das die ganze Persönlichkeit betrifft, das sich das ganze Leben lang ausbaut und sich über ein Hinausgehen über die eigene Ich-Bezogenheit erstreckt. \textquotedblleft Glück dient zur Vermittlung von Harmonie mit sich selbst und der Welt, Werthaftigkeit des Lebens und Gewinn neuer Gefühle und Erfahrungen, es kann vor Depressionen schützen.\textquotedblright \ \cite{Mayring:1991} \newline
Veenhoven \cite{Veenhoven:1991} verwendet das Wort Glück (happiness) gleichbedeutend für Lebenszufriedenheit (life satisfaction) und beschreibt das Ausmass an Qualität, welches das Leben als Ganzes für ein Individuum bereit stellt. Vereinfacht gesagt, wie sehr ein Individuum das von ihm geführte Leben mag. Veenhoven unterscheidet zwischen einem affektiven und einem kognitiven Aspekt, was das Leben als Ganzes betrifft. Der affektive Aspekt betrifft das Niveau der Lust, resp. wie gut sich eine Person im Allgemeinen fühlt  und der kognitive Aspekt steht für den Grad der Zufriedenheit mit dem, was die Person denkt im Leben erreicht zu haben.\newline
Die Unterscheidung gemäss Becker \cite{Becker:1994} in \gls{aw}, das die aktuelle Befindlichkeit eines Menschen charakterisiert und \gls{hw}, welches als relativ stabile Eigenschaft eines Menschen zu verstehen ist. Becker verwendet \gls{aw} als Oberbegriff für die Beschreibung des momentanen Erlebens einer Person. Dieses beinhaltet positiv getönte Gefühle, Stimmungen und körperliche Empfindungen sowie das Fehlen von Beschwerden. Im Gegensatz zum \gls{aw} werden mit dem \gls{hw} Urteile über das für eine Person typische Wohlbefinden charakterisiert. Unter Urteile sind die Aussagen einer Person über \gls{hw} gemeint, die primär durch kognitive Prozesse zustande kommen.

%UK-Begriffsabgrenzung
\section{Begriffsabgrenzung}\label{abgrenzung}
Aus der Begriffserklärung geht hervor, dass es verschiedene Möglichkeiten gibt \gls{swb} zu beschreiben und zu erklären. Wird eine weitere Art der Begriffserklärung hinzugenommen, die lexikalische, so geht daraus hervor, dass die deutsche Sprache eine gewisse Doppeldeutigkeit zum Begriff Glück bietet \cite{Mayring:1991}. Viele andere Sprachen unterscheiden zwischen Glück im Sinne von Zufall und Glück im Sinne von Erfüllung. Diese Unterscheidung macht die deutsche Sprache nicht. Wenn in dieser Arbeit die Rede von Glück im Sinne von \gls{swb} ist, so wird immer die Bedeutung im Sinne der Erfüllung gemeint. Glück im Sinne des Zufalls ist psychologisch und somit für diese Arbeit nicht relevant.

%UK-Historischer Überblick
\section{Historischer Überblick und Entstehung}\label{überblick}
An dieser Stelle soll ein Exkurs in die Philosophie vorgenommen werden, da Glück von Anbeginn philosophisch behandelt wurde und als ein zentraler Begriff in vielen philosophischen Systemen gilt \cite{Mayring:1991}. Eine Beschränkung erfolgt auf die abendländische Philosophie. Einerseits weil die abendländische Philosophie für unser Denken bestimmend gewesen ist und andererseits weil eine ausführliche Beschäftigung mit den restlichen Denkweisen, wie chinesische oder indische Philosophie, den Rahmen dieser Arbeit sprengen würde. \newline
In der vor-philosophischen Epoche des alten Griechenlands galt Glück gemäss Mayring \cite{Mayring:1991} als eine Gabe der Götter. Diese Gabe wurde als Reichtum, Ehre, Macht und Gesundheit symbolisiert. Erst der Vorsokratiker Demokrit (ca. 470 bis ca. 370 v. Chr.) beschäftigte sich mit dem Thema, was der Mensch tun könne, um glücklich zu sein. Demokrit verwies darauf, dass Glück \textquotedblleft darüber hinaus und in stärkerem Mass von der inneren Verfassung des Menschen abhängt \textquotedblright \cite{Becker:1994} und nicht mehr durch äussere Güter, sondern durch die innere Haltung begründet ist \cite{Mayring:1991}. Diese Lehre wurde bestimmend für die griechische Lehre der Glückseligkeit und wurde in den Mittelpunkt der platonischen Philosophie des tugendhaften guten Lebens gestellt \citeyear<ebda.,>{Mayring:1991}.\newline
Aristoteles (384-322 v. Chr.), der gemäss \citeA{Mayring:2003} als wohl erster ein Lehrbuch über die Psychologie (\textquotedblleft über die Seele\textquotedblright) geschrieben hat, behandelt die darin beschreibenden Emotionen auf zwei Ebenen, der Ebene der Tugend und der Ebene der lust- und unlustverbundenen Passionen. Passionen sind Begleiterscheinungen der Tätigkeit (Affekte), wobei Tugenden direkt willentlich angestrebt werden und in Form von Glück dominiert werden. Demzufolge wird Glück mit dem tugendhaften Leben gleichgesetzt, einem Leben in Tätigkeit, in sozialen und politischen Bezügen, in wissenschaftlichem und künstlerischem Engagement und in der Entfaltung der eigenen Fähigkeit \citeyear<ebda.,>{Mayring:2003}. Aristoteles hat in seiner Lehre die Emotionen in aktuelle Gefühlszustände und in Persönlichkeitseigenschaften eingeteilt, eine Konzeption, die heute unter dem Begriff \textquotedblleft State-Trate-Ansatz\textquotedblright \ in der Psychologie verwendet wird. Becker \cite{Becker:1994} deutet darauf hin, dass die griechische Glückskonzeption von Epikur (341-270 v.Chr.) und von den Stoikern auf eine strenge Lebensführung und Kontrolle aller Affekte hinauslaufen. Also immer im Zusammenhang mit mehr oder weniger vernunftgesteuerten Handlungen, aber auch als Lust und Unlust spendend \cite{Mayring:2003}.\newline
Die mittelalterliche Philosophie steht im Gegensatz zum eher vernunftbetonten griechischen Denken und ist dogmatisch vom christlichen Glauben bestimmt \cite{Mayring:2003}. Die weltlichen Affekte galten zu Zeiten von Clemens von Alexandria (150-211 n.Chr.) als Dämonen. Emotionen wie Lust und Unlust gehörten den menschlichen Schwächen an wie Furcht, Neid, körperliche Lust, fleischlicher Appetit. Daneben galten Barmherzigkeit, Zorn und Mitleid zu den göttlichen Emotionen. Die Hauptquelle für diese theologisch geprägte Philosophie war die heilige Schrift. Gemäss Mayring \cite{Mayring:1991} ging es den Christen also nicht um irdisches Glück, sondern um himmlisches Heil. Erst bei Thomas von Aquin (1225-1279) ist ein leichter Wandel zu erkennen. Sein Denken war stark von Aristoteles geprägt. Auf dem Höhepunkt der Scholastik setzte er die Liebe ins Zentrum der Grundprinzipien, die vom Willen abhängig waren. Wobei der Verstand die Liebe und den Willen zur Liebe kontrollieren kann. Doch auch er war der Meinung, dass \textquotedblleft das letzte Glück des Menschen in diesem Leben nicht zu finden sei\textquotedblright, \cite{Becker:1994}. \newline
In der Philosophie der Neuzeit zeigt sich wieder eine stärkere Hinwendung zur antiken griechischen Glücksphilosophie \cite{Mayring:1991}. Eine Neubestimmung des Glücksbegriffs wurde in der Aufklärung durch John Locke (1632-1704) vorgenommen. Das Handeln des Menschen wird durch die Begierde nach Vergnügen bestimmt, wobei das Gute durch dieses Handeln entsteht. Diese Überlegungen flossen in Jeremy Benthams (1748-1832) hedonistisches Kalkül, welches besagt, dass alles persönliche und politische Handeln dazu dienen soll, das grösstmögliche Glück für die grösstmögliche Zahl von Menschen zu erreichen \citeyear<ebda.,>{Mayring:1991}. Immanuel Kant (1724-1804) meinte, dass Glück vom Handelnden nicht direkt zu erreichen ist, da dieser allwissend sein müsste. Der Mensch könne sich nur durch moralisches Handeln für Glück würdig erweisen. Im 18. und 19. Jahrhundert wich die reine Glückslehre einer materialistischen Philosophie der Gesellschaftskritik. Glück wird als Menschenrecht deklariert und wird damit zur Parole des politischen Kampfes gegen die herrschenden Mächte \cite{Jones:1953}. Glück wird zum Sinn des Lebens, auf eine auf Vernunft gebaute sittliche Kategorie. \newline
In der Philosophie der Gegenwart wird gemäss Becker \cite{Becker:1994} Glück als Ergebnis gelungener Selbstverwirklichung thematisiert \cite{Kambartel:1978}. Gemäss dem Wiener Nervenarzt und Philosoph Frankl \cite{Frankl:1976}, der sich mit dem Thema Selbstverwirklichung auseinandersetzte, ist der Mensch weniger dem Glück als dem Sinn interessiert. Diesen Sinn findet der Mensch durch Selbsttranszendenz, in dem er sich anderen Personen und Aufgaben widmet und über sich hinaus geht. Tartakiewicz \cite{Tartakiewicz:1984} schrieb in seiner Monographie über das Glück, dass persönliche Voraussetzungen und unterschiedliche äussere Lebensbedingungen eines Menschen unterschiedliche Wege zum Glück eröffnen. \newline
Zusammenfassend lässt sich aus dieser knappen und notwendigerweise kurzen Aufstellung entnehmen, dass der heutige Begriff des \gls{swb} und dem dafür synonym verwendeten Begriff \textquotedblleft Glück\textquotedblright \ bis in die klassische Antike zurückreichenden Philosophie gründet und somit prägend für die folgende Entwicklung war. Trotz mancher Gegensätze lassen sich auch einheitliche Glückstheorien finden wie z.B.: dass das Streben nach Glück universell und jedem Menschen eigen ist \cite{Becker:1994}. 
  
%UK-SWB in der positiven Psychologie
\section{Verwendung in der Positiven Psychologie}\label{sec.swbPospsy}
Die Positive Psychologie ist einerseits ein wissenschaftliches und andererseits ein klinisches Vorhaben \cite{Carr:2011}. Die wissenschaftliche Auseinandersetzung innerhalb der Positiven Psychologie befasst sich mit dem Verstehen und dem Verbessern von positiven Aspekten im Leben:
\begin{itemize}
\item Glück und Wohlbefinden,
\item positiven Eigenschaften und 
\item der Entwicklung von positiven Beziehungen, sozialen Systemen und sozialen Einrichtungen \cite{Lopez:2009,Seligman:2002}.
\end{itemize}
Die positive Psychologie beschäftigt sich mit dem angenehmen Leben (engl. pleasant life), dem engagierten Leben (engl. engaged life) und dem bedeutungsvollen Leben (engl. meaningful life). Diese drei Faktoren führen zum Glück und sind für das Wohlbefinden zuständig \cite{Carr:2011}. In einer Studie von \citeA{Peterson:2005} mit 845 erwachsenen Personen wurde festgestellt, dass diese drei Aspekte über Vergnügen, Engagement und sinnvolles Handeln zu Lebenszufriedenheit führen. Zudem wurde die Erkenntnis gewonnen, dass Engagement und sinnvolles Handeln stärker mit der Lebenszufriedenheit einhergeht, als über das reine Vergnügen. Das klinische Bestreben der Positiven Psychologie hat nicht das Korrigieren von Defiziten zum Ziel, sondern die Steigerung von Wohlbefinden und Glück \cite{Carr:2011}. Vielmehr soll sie zur Vervollständigung und nicht zum vollständigen Ersatz der Klinischen Psychologie beitragen.
%SubSec Postivie Traits
\subsection{Positive Emotionen}\label{subsec.swbPosEmotion}
Der Begründer der Positiven Psychologie, Professor Martin Seligman \citeyear{Seligman:2002}, klassifizierte positive Emotionen in drei Kategorien: diejenigen Emotionen, die mit der Vergangenheit verknüpft sind, solche, die mit der Gegenwart einhergehen und Emotionen, die mit der Zukunft zusammenhängen. Positive Emotionen, die mit der Zukunft einher gehen sind Optimismus, Hoffnung, Zuversicht, Glaube und Vertrauen. Zufriedenheit, Behagen, Erfüllung, Stolz und Gelassenheit sind Emotionen, die mit der Vergangenheit einhergehen. Seligman unterteilt die positiven Emotionen in der Gegenwart in zwei Klassen: Momentanes Vergnügen und eher überdauernde Befriedigung. Momentane Vergnügen werden unterteilt in körperliche Vergnügen und höhere Vergnügen. Körperlichen Vergnügen werden über die Sinne hervorgebracht, wie Sex, angenehme Gerüche und delikaten Speisen. Höhere Vergnügen entstehen über komplexe Aktivitäten und beinhalten Emotionen wie Seligkeit, Fröhlichkeit, Trost, Ekstase und Überschwänglichkeit. Überdauernde Befriedigung unterscheidet sich insofern vom momentanen Vergnügen als dass sie sich in Zeiten der Absorbation oder dem \textquotedblleft Flow\textquotedblright \ einstellt. Zustände also, die durch ein erhöhtes Engagement erreicht werden \cite{Carr:2011}. Segeln, unterrichten von anderen Personen oder bei den Vorbereitungen auf eine Prüfung helfen, sind solche Zustände. 

%SubSec Postivie Traits
\subsection{Messen von Glück}\label{subsec.swbMeasuringHappiness}
Wie glücklich sind die meisten Menschen? Dieser Frage ist Professor Ed Diener \citeyear{Myers:1995} mit der Befragung von über einer Million Menschen nachgegangen. Er transformierte die Daten in eine Skala von 0 bis 10, wobei 10 für \textquotedblleft extrem Glücklich\textquotedblright \ und 0 für \textquotedblleft sehr unglücklich\textquotedblright \ steht. Dabei fand er heraus, dass der Durchschnitt der untersuchten Personen mässig glücklich war. \newline
In Studien zu Glück und \gls{swb} sind viele Konstrukte für deren Messung zu Einsatz gekommen. In den meisten Studien wurde mittels einfachem Fragebogen gemessen \cite{Carr:2011}, in denen die Antworten mittels 5-, 7- oder 10 Punktskalen gemessen wurden. \citeA{Fordyce:1988} entwickelte eine 2-Item-Skala, die einerseits das generelle Niveau des \gls{swb} befragt und andererseits die durchschnittliche Zeitspanne, in der die befragte Person sich glücklich fühlt. Anspruchsvollere Mehrfach-Item-Skalen mit guten Werten in Reliabilität und Validität wurden entwickelt, so wie die Skala von \citeA{Diener:1985}, die die Lebenszufriedenheit misst (siehe einleitendes Beispiel). Diese Skala findet nach wie vor weite Verbreitung in Amerika \cite{Carr:2011}. Das sogenannte \textquotedblleft Oxford Happiness Questionnaire\textquotedblright \ von \citeA{Hills:2002} findet vor allem in England weite Verbreitung. Weitere Tests wie der \textquotedblleft Warwick-Edinburgh Mental Well-Being Scale\textquotedblright-Test von \citeA{Tennant:2007} oder der \textquotedblleft Bipolar depression-happiness scale\textquotedblright \ Test von \citeA{Joseph:1998} sind weitere verbreitete Tests, die eine starke psychometrische Ausprägung aufweisen.\newline
Während Tests mit Einfach-Item- und Mehrfach-Item-Skalen die globale Sichtweise auf das \gls{swb} von Personen messen, bieten die \textquotedblleft Experience Sampling Methods (ESM)\textquotedblright \ Möglichkeiten einer Zeitpunkt-Messung an \cite{Hektner:2007}. In dieser Methode werden die Versuchspersonen mit einem Pager (Gerät zur Nachrichtenübermittlung) versehen, den sie während der ganzen Testzeit bei sich tragen und der die Person von Zeit zu Zeit auffordert, ihre aktuellen Gefühlszustände zu notieren. Im Gegensatz zu den Einfach-Item- und Mehrfach-Item-Tests, die für die Beurteilung des \gls{swb} während einer langen Zeitperioden eingesetzt werden, sind die ESM Tests vor allem hilfreich in der Erfassung von \gls{swb}-Schwankungen in einer relativ kurzen Zeitperiode.

%SubSec Postivie Traits
\subsection{Persönliche Eigenschaften}\label{subsec.swbPersTraits}
Anhand umfänglichen Beurteilungen von Test und Meta-Analysen stellte sich heraus, dass glückliche und unglückliche Menschen unterschiedliche persönliche Profile aufweisen \cite{Diener:1999, Steel:2008}. In westlichen Kulturen sind glückliche Personen tendenziell eher extravertiert, stabil, gewissenhaft, liebenswürdig, optimistisch und haben einen hohen Selbstwert. Im Gegenzug zeigen unglückliche Personen eher einen hohen Wert an Neurotizismus, sind introvertiert und zeigen einen niedrigen Wert an Gewissenhaftigkeit und Liebenswürdigkeit. \newline
Verschiedene Faktoren weisen auf den Zusammenhang zwischen Extraversion und \gls{swb} hin \cite{Diener:1999}: Extravertierte Menschen fügen sich besser in die sozialen Anforderungen ein, wobei diese Personen eher in eine soziale Interaktion gelangen. Sie befinden sich daher vermehrt in Situationen, die ihnen entsprechen und fühlen sich dadurch glücklicher. Weitere Anzeichen lassen darauf schliessen, dass extravertierte und neurotische Personen mehr positive und negative Ereignisse erfahren \cite{Carr:2011}. Personen mit einem hohen Wert an Extraversion befinden sich öfter in positiven Situationen und erhalten so mehr \gls{swb}. Personen mit einem hohen Wert an Neurotizismus befinden sich eher in negativen Situationen, was sich negativ auf das \gls{swb} auswirkt.	


