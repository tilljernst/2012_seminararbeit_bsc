%%%%%%%%%%%%%%%%%%%%%%%%%%%%%%%%%%%%%%%%%%%%%%%%%%%%%%%%%%%%%%%%%
%_____________ ___    _____  __      __ 
%\____    /   |   \  /  _  \/  \    /  \  Institute of Applied
%  /     /    ~    \/  /_\  \   \/\/   /  Psychology
% /     /\    Y    /    |    \        /   Zürcher Hochschule 
%/_______ \___|_  /\____|__  /\__/\  /    fuer Angewandte Wissen.
%        \/     \/         \/      \/                           
%%%%%%%%%%%%%%%%%%%%%%%%%%%%%%%%%%%%%%%%%%%%%%%%%%%%%%%%%%%%%%%%%
%
% Project     : Seminararbeit
% Title       : 
% File        : subjektivesWohlbefinden.tex Rev. 00
% Date        : 10.10.2012
% Author      : Till J. Ernst
%
%%%%%%%%%%%%%%%%%%%%%%%%%%%%%%%%%%%%%%%%%%%%%%%%%%%%%%%%%%%%%%%%%
\chapter{Subjektives Wohlbefinden}\label{chap.swb}
TBD HK-Subjektives Wohlbefinden
\citeA{Seligman:2003}\newline 
Seligman \citeyear{Seligman:2003}\newline 
\cite<e.g.,>{Meyen:2010}\newline
Dieses Kapitel befasst sich mit der Begriffserklärung und der Abgrenzung des Begriffs \textit{\gls{swb}}. Des Weiteren wird auf die historische Entstehung, die Verwendung in der positiven Psychologie und auf das psychologische Konstrukt des \gls{swb} eingegangen.
%UK-Begriffserklärung
\section{Begriffserklärung}\label{begriff}
\glsreset{swb}
Der Begriff \textit{Wohlbefinden} und mit ihm der zusammengesetzte Begriff \textit{Subjektives Wohlbefinden} wird in der Literatur sehr unterschiedlich beschrieben. Eine eindeutige Erklärung ist somit kaum möglich. Begriffe wie z.B. Glück, Glücksempfinden, Lebenszufriedenheit, Lebensqualität, positive Emotionen, positive Affekte u.v.a., werden teilweise identisch behandelt, teilweise aber auch voneinander abgegrenzt. Um den Begriff \textit{\gls{swb}} zu beschreiben und zu erklären, sind unterschiedliche Vorgehensweisen denkbar. Es könnte die Herkunft und die historische Entwicklung des Begriffs in der Sprache zugezogen werden (\textit{etymologische} Erklärung). Den heutigen Wortgebrauch, wie er in den Lexika festgelegt ist (\textit{lexikalische} Erklärung) oder man könnte Menschen dazu befragen, was sie unter dem Begriff verstehen (\textit{empirische} Erklärung). 
Im Folgenden werden einige Begriffe anhand bisheriger Konzeptionen und Theorieansätzen erläutert, um Transparenz in die unterschiedliche Verwendung zu bringen und um ein für diese Arbeit notwendiges Wissen zu vermitteln, wie das \gls{swb} ind den folgenden Kapiteln zu verstehen ist.\newline
Gemäss Seligman  \cite{Seligman:2003} sind die Worte \textit{Glück} oder \textit{Glücklichkeit} und \textit{Wohlbefinden} austauschbar und umfassen positive Gefühle, wie Ekstase oder Behagen und positive Verhaltensweisen, die nicht über eine Gefühlskomponente verfügen, wie Absorption oder Engagement. Demzufolge lässt sich daraus schliessen, dass Glück und Wohlbefinden sich manchmal auf Gefühle, manchmal auf Aktivitäten, in denen nichts gefühlt wird, beziehen.\newline
Mayring \cite{Mayring:1991} versteht unter Glück ein Erleben auf einer emotionaler, kognitiver und aktionaler Ebene, das die ganze Persönlichkeit betrifft, das sich das ganze Leben lang ausbaut und sich über ein Hinausgehen über die eigene Ich-Bezogenheit erstreckt. \textquotedblleft Glück dient zur Vermittlung von Harmonie mit sich selbst und der Welt, Werthaftigkeit des Lebens und Gewinn neuer Gefühle und Erfahrungen, es kann vor Depressionen schützen.\textquotedblright \quad \cite[S.179]{Mayring:1991} \newline
Veenhoven \cite{Veenhoven:1991} verwendet das Wort Glück (\textit{happiness}) gleichbedeutend für Lebenszufriedenheit (\textit{life satisfaction}) und beschreibt das Ausmass an Qualität, was das Leben als Ganzes für ein Individuum bereit stellt. Vereinfacht gesagt, wie sehr ein Individuum das von ihm geführte Leben mag. Veenhoven unterscheidet zwischen einem affektiven und einem kognitiven Aspekt, was das Leben als Ganzes betrifft. Der affektive Aspekt betrifft das Niveau der Lust, resp. wie gut sich eine Person im Allgemeinen fühlt  und der kognitive Aspekt steht für den Grad der Zufriedenheit mit dem, was die Person denkt im Leben erreicht zu haben.\newline
Becker \cite{Becker:1994} unterscheidet zwischen \textit{\gls{aw}}, das die aktuelle Befindlichkeit eines Menschen charakterisiert und \textit{\gls{hw}}, welches als relativ stabile Eigenschaft eines Menschen zu verstehen ist. Becker verwendet \gls{aw} als Oberbegriff für die Beschreibung des momentanen Erlebens einer Person. Dieses beinhaltet positive getönte Gefühle, Stimmungen und körperliche Empfindungen sowie das Fehlen von Beschwerden. Im Gegensatz zum \gls{aw} werden mit dem \gls{hw} \textit{Urteile} über das für eine Person typische Wohlbefinden charakterisiert. Unter Urteile sind die Aussagen einer Person über \gls{hw} gemeint, die primär durch kognitive Prozesse zustande kommen.

%UK-Begriffsabgrenzung
\section{Begriffsabgrenzung}\label{abgrenzung}
Aus der Begriffserklärung geht hervor, dass es verschiedene Möglichkeiten gibt \textit{\gls{swb}} zu beschreiben und zu erklären. Wird eine weitere Art der Begriffserklärung hinzugenommen, die lexikalische, so geht daraus hervor, dass die deutsche Sprache eine gewisse Doppeldeutigkeit zum Begriff \textit{Glück} bietet (\cite{Mayring:1991}). Viele andere Sprachen unterscheiden zwischen \textit{Glück} im Sinne von Zufall und \textit{Glück} im Sinne von Erfüllung. Diese Unterscheidung macht die deutsche Sprache nicht. Wenn in dieser Arbeit die Rede von \textit{Glück} im Sinne von \textit{\gls{swb}} ist, so wird immer die Bedeutung im Sinne der Erfüllung gemeint. \textit{Glück} im Sinne des Zufalls ist psychologisch und somit für diese Arbeit nicht relevant.

%UK-Historischer Überblick
\section{Historischer Überblick und Entstehung}\label{überblick}
\cite{Mayring:1991}
An dieser Stelle soll ein Ausflug in die Philosophie vorgenommen werden, da Glück von Anbeginn der philosphisch behandelt wurde und als ein zentraler Begriff in vieler philosophischer Systemen gilt \cite{Mayring:1991}. Eine Beschränkung erfolgt auf die abendländische Philosophie. Einerseits weil die abendländische Philosophie für unser Denken bestimmend gewesen ist und andererseits weil eine ausführliche Beschäftigung mit den restlichen Denkweisen, wie chinesische oder indische Philosophie, den Rahmen dieser Arbeit sprengen würde. \newline
In der vorphilosophischen Epoche des alten Griechenlands galt Glück gemäss Mayring \cite{Mayring:1991} als eine Gabe der Götter. Diese Gabe wurde als Reichtum, Ehre, Macht und Gesundheit symbolisiert. Erst der vorsokratiker Demokrit (ca. 470 bis ca. 370 v. Chr.) beschäftigte sich mit dem Thema, was der Mensch tun könne, um glücklich zu sein. Demokrit verwies darauf, dass Glück \textquotedblleft darüber hinaus und in stärkerem Mass von der inneren Verfassung des Menschen abhängt \textquotedblright \cite{Becker:1994} und nicht mehr durch äussere Güter, sondern durch die innere Haltung begründet ist \cite{Mayring:1991}. Diese Lehre wurde bestimmend für die griechische Lehre der Glückseligkeit und wurde in den Mittelpunkt der platonischen Philospohie des tugendhaften guten Lebens gestellt (ebda., 1991).\newline
Aristoteles (384-322 v. Chr.), der gemäss \cite{Mayring:2003} als wohl erster ein Lehrbuch über die Psychologie (\textquotedblleft über die Seele \textquotedblright) geschrieben hat, behandelt die darin beschreibenden Emotionen auf zwei Ebenen, der Ebene der Tugend und der Ebene der lus- und unlustverbundenen Passionen. Passionen sind Begleiterscheinungen der Tätigkeit (Affekte), wobei Tugenden direkt willentlich angestrebt werden und in Form von Glück dominiert werden. Demzufolge wird Glück mit dem tugendhaften Leben gleichgesetzt, einem Leben in Tätigkeit, in sozialen und politischen Bezügen, in wissenschaftlichen und künstlerischen Engagement und in der Entfaltung der eigenen Fähigkeit (ebda, 2003). Aristoteles hat in seiner Lehre die Emotionen in aktuelle Gefühlszustände und ind Persönlichkeitseigenschaften eingeteilt, eine Konzeption, die heute unter dem Begriff \textquotedblleft State-Trate-Ansatz\textquotedblright in der Psychologie verwendet wird (siehe \ref{konzept} - \nameref{konzept} auf Seite \pageref{konzept}). Becker \cite{Becker:1994} deutet darauf hin, dass die griechische Glückskonzeptionen von Epikur (341-270 v.Chr.) und von den Stoikern auf eine strenge Lebensführung und Kontrolle aller Affekte hinauslaufen. Also immer im Zusammenhang mit mehr oder weniger vernunftgesteuerten Handlungen, aber auch als Lust und Unlust spendend \cite{Mayring:2003}.\newline
Die mittelalterlichen Philosophie steht im Gegensatz zum eher vernunftbetonten griechischen Denken und ist dogmatisch vom christlichen Glauben bestimmt \cite{Mayring:2003}. Die Weltlichen Affekte galten zu Zeiten von Clemens von Alexandria (150-211 n.Chr.) als Dämonen bezeichnet. Emotionen wie Lust und Unlust gehörten den menschlichen Schwächen an wie Furch, Neid, körperliche Lust, fleischlicher Appetit. Daneben galt Barmherzigkeit, Zorn und Mitleid zu göttlichen Emotionen. Die Hauptquelle für diese theologisch geprägte Philosophie war die heilige Schrift. Gemäss Mayring \cite{Mayring:1991} ging es den Christen also nicht um irdisches Glück, sondern um himmlisches Heil. Erst bei Thomas von Aquin (1225-1279) ist ein leichter Wandel zu erkennen. Sein Denken war stark von Aristoteles geprägt. Auf dem Höhepunkt der Scholastik setzte er die Liebe ins Zentrum der vom Willen abhängigen Grundprinzip. Wobei der Verstand die Liebe und den Willen zur Liebe kontrollieren kann. Doch auch er war der Meinung, dass \textquotedblleft das letzte Glück des Menschen in diesem Leben nicht zu finden sei \textquotedblright, \cite{Becker:1994}. \newline
In der Philosophie der Neuzeit zeigt sich eine wieder stärkere Hinwendung zur antiken griechischen Glücksphilosophie \cite{Mayring:1991}. Eine Neubestimmung des Glücksbegriffs wurde in der Aufklärung durch John Locke (1632-1704) vorgenommen. Das Handeln des Menschen wird durch die Begierde nach Vergnügen bestimmt, wobei das Gute durch dieses Handeln entsteht. Diese Überlegungen flossen in Jeremy Benthams (1748-1832) hedonistisches Kalkül, welches besagt, dass alles persönliche und politische Handeln dazu dienen soll, das grösstmögliche Glück für die grösstmögliche Zahl von Menschen zu erreichen (ebda, 1991). Immanuel Kant (1724-1804) meinte, Glück vom Handelnden nicht direkt zu erreichen ist, da dieser allwissend sein müsste. Der Mensch könne sich nur durch moralisches Handeln für Glück würdig erweisen. Im 18. und 19. Jahrhundert wich die reine Glückslehre einer materialistischen Philosophie der Gesellschaftskritik. Glück wird als Menschenrecht deklariert und wird damit zur Parole des politischen Kampfes gegen die herrschenden Mächte \cite{Jones:1953}. Glück wird zum Sinn des Lebens, auf eine auf Vernunft gebaute sittliche Kategorie. \newline
In der Philosophie der Gegenwart wird gemäss Becker \cite{Becker:1994} Glück als Ergebnis gelungener Selbstverwirklichung thematisiert \cite{Kambartel:1978}. Gemäss dem Wiener Nervenarzt und Philosoph Frankl \cite{Frankl:1976}, der sich mit dem Thema Selbstverwirklichung auseinander setzte, ist der Mensch weniger dem Glück als dem Sinn interessiert. Diesen Sinn findet der Mensch durch Selbsttranszendenz, in dem er sich anderen Personen und Aufgaben widmet und über sich hinaus geht. Tartakiewicz \cite{Tartakiewicz:1984} schrieb in seiner Monographie über das Glück, dass persönliche Voraussetzungen und unterschiedliche äussere Lebensbedingungen eines Menschen, unterschiedliche Wege zum Glück eröffnen. \newline
Zusammenfassend lässt sich aus diesen knappen und notwendigerweise kurzen Zusammenstellung entnehmen, dass der heutige Begriff des \gls{swb} und dem dafür synonym verwendeten Begriff \textquotedblleft Glück\textquotedblright  bis in die klassische Antike zurückreichenden Philosophie gründet und somit prägend für die folgende Entwicklung war. Trotz mancher Gegensätze lassen sich auch einheitliche Glückstheorien finden wie z.B.: dass das Streben nach Glück universell und jedem Menschen eigen ist. 
  
%UK-SWB in der positiven Psychologie
\section{Verwendung in der positiven Psychologie}\label{pospsy}
TBD

%UK-Psychologisches Konzept
\section{Psychologische Konzepte}\label{konzept}
TBD

\subsection{Positive Traits}\label{traits}
TBD Big-Five (Neurotizismus und Extraversion).


