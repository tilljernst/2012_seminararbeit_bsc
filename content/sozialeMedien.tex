%%%%%%%%%%%%%%%%%%%%%%%%%%%%%%%%%%%%%%%%%%%%%%%%%%%%%%%%%%%%%%%%%
%_____________ ___    _____  __      __ 
%\____    /   |   \  /  _  \/  \    /  \  Institute of Applied
%  /     /    ~    \/  /_\  \   \/\/   /  Psychology
% /     /\    Y    /    |    \        /   Zuercher Hochschule 
%/_______ \___|_  /\____|__  /\__/\  /    fuer Angewandte Wissen.
%        \/     \/         \/      \/                           
%%%%%%%%%%%%%%%%%%%%%%%%%%%%%%%%%%%%%%%%%%%%%%%%%%%%%%%%%%%%%%%%%
%
% Project     : Seminararbeit
% Title       : 
% File        : sozialeMedien.tex Rev. 00
% Date        : 10.10.2012
% Author      : Till J. Ernst
%
%%%%%%%%%%%%%%%%%%%%%%%%%%%%%%%%%%%%%%%%%%%%%%%%%%%%%%%%%%%%%%%%%

\chapter{Soziale Medien}\label{chap.sm}
\glsresetall
\textit{Ein paar reisserische Zahlen für den Einstieg wäre nicht schlecht...}\par 

In einem ersten Teil wird auf die Begrifflichkeit von \gls{sm} eingegangen. Danach folgt eine Klassifikation und schliesst mit einem kurzen Blick in die Zukunft.\newline
Da es sich beim Thema \gls{sm} um eine Technologie handelt, die vorwiegend in englischer Sprache verfasst und definiert ist, kann durch fehlende Übersetzung die gender-gerechte Form nicht immer berücksichtigt werden. Wo immer ein Rückschluss auf eine männliche Form vorgenommen werden kann, ist implizit auch die weibliche Form gemeint und umgekehrt.
%UK Zentrale Begriffe
\section{Begriffserklärung}\label{sec.begriff}
Gemäss \citeA{Sjurts:2011}, ist \gls{sm} (engl. social media) ein Sammelbegriff, für internet-basierte mediale Angebote, die auf einer sozialen Interaktionen basieren. Die Angebote, auch Applikationen genannt, bauen auf den Ideologien und den technischen Möglichkeiten von Web 2.0 auf \cite{Kaplan:2010} und ermöglichen das Erstellen und Austauschen von nutzererzeugten Inhalten (engl. user generated content).\newline
Web 2.0 und nutzerzeugte Inhalte bilden dabei die Grundkonzepte von \gls{sm}. Der Begriff und die Bedeutung Web 2.0 wurde 2004 neu eingeführt. Ziel war es, die Nutzung des Weltweiten Netzes (engl. World Wide Web, kurz Web oder WWW) zwischen Personen die Software entwickeln und Personen die sie nutzen, neu zu definieren (ebda.,2010). Web 2.0 stellt dabei eine Plattform dar, auf der nicht mehr einzelne Personen für den Inhalt und die Verteilung von Angebote verantwortlich waren, sondern eine Vielzahl von beteiligten Personen gemeinsam. Neben dieser Plattform, die als technologisches Fundament gilt, kann der nutzerzeugte Inhalt als die Summe aller Möglichkeiten betrachtet werden, wie die beteiligten Personen die \gls{sm} nutzen können. Dazu müssen gemäss \citeA{OECD:2007} drei grundlegende Anforderungen erfüllt sein: Erstens müssen nutzererzeugte Inhalte auf einer öffentlich publizierten Webseite zugänglich sein, oder auf einer \gls{sns} für eine ausgewählte Gruppe zur Verfügung stehen, was z.B. Mail ausschliesst. Zweitens muss der Inhalt einem gewissen Anteil an kreativen Aufwand genügen, was eine reine Kopie eines bestehenden Inhalts bereits erfüllt und drittens muss der Inhalt ausserhalb eines professionellen Umfeldes erzeugt worden sein, auf dem Hintergrund eines kommerziellen Marktes. Nutzerzeugte Inhalte wurden bereits vor dem Web 2.0, in den frühen 1980 beschrieben. Aber erst mit den Möglichkeiten von Web 2.0, technologischer Treiber (z.B. Breitband-Internetanschluss), öconomischer Veränderung (z.B. frei zugängliche Angebote für die Generierung von Inhalten) und soziologischer Faktoren (z.B. Entstehung einer Generation mit erheblichem technischen Fachwissen, vgl. \textquotedblleft digital natives \textquotedblright) konnten die gesamten Möglichkeiten angewendet werden und umschreiben den heutigen Begriff \gls{sm}.\newline 
Eine weitere ergänzende Definition gemäss \citeA{Ahlqvist:2008} unterteilt \gls{sm} in drei Hauptkategorien: Inhalt, Gemeinschaft und Web 2.0. Inhalt bezeichnet den von Personen generierte Inhalt, der unterschiedlicher Art sein kann (z.B. Bilder, Fotos, Videos, Kommentare, Standortinformationen, etc.). Gemeinschaft ist bereits im englischen Wort \textquotedblleft social\textquotedblright{} enthalten und bezeichnet das Bereitstellen von Inhalt auf dem Web und das Interesse am Inhalt Anderer. \gls{sm} wird dazu verwendet direkt oder mittels aufgezeichneten Daten miteinander zu kommunizieren. Die Webtechnologien und die Anwendungen, die es Personen ermöglicht über das Web zu kommunizieren, bezeichnet die dritte Kategorie Web 2.0. \newline
\gls{sm} vereint somit mobile und web-basierte Technik, um nutzererzeugten Inhalt auf einer flexiblen Plattform zwischen Einzelpersonen und Gruppen zu teilen, zu erzeugen, untereinander zu bewerten und zu verändern \cite{Kietzmann:2011}.
 
%UK Arten von Sozialen Medien
\section{Klassifizierung}\label{sec.klassifiezierung}
\gls{sm} beinhalten eine riesige Menge an Anwendungen und Möglichkeiten. Eine Klassifizierung kann dabei helfen, den Überblick zu bewahren. \citeA{Kietzmann:2011} teilt die \gls{sm} in separate Blöcke ein, die sich bienenwabenartig aneinanderreihen lassen. Die funktionalen Blöcke setzen sich zusammen aus Identität, Konversation, gemeinsame Nutzung, Anwesenheit, Beziehung, Ansehen und Gruppe. Jeder Block untersucht dabei eine spezifische Facette einer Person und erlaubt die Funktionen von \gls{sm} differenziert in verschiedene Ebenen zu unterteilen.\newline
Das Klassifizierungsschema von \citeA{Kaplan:2010} basiert auf medienwissenschaftlichen und soziologischen Theorien. Aus medienwissenschaftlicher Sicht spielt die Theorie der sozialen Präsenz (engl. social presence) von \citeA{Short:1976} und die Medienreichhaltigkeitstheorie (engl. media richness) von \citeA{Daft:1986} ein Rolle. Die Theorie der sozialen Präsenz geht davon aus, dass je höher die soziale Präsenz eines Kommunikationsmediums ist, desto höher ist der gegenseitige Einfluss auf die Kommunikationspartner und Kommunikationspartnerinnen. Medienreichhaltigkeitstheorien nach \citeA{Daft:1986} basieren auf der Annahme, dass das Hauptziel jeglicher Kommunikation, die Auflösung von Mehrdeutigkeit und die Reduktion von Unsicherheit sein sollte. Im Hinblick auf die soziologischen Theorien, die eine wichtige Rolle im Bereich der \gls{sm} spielen, ist das Konzept der Selbstdarstellung (engl. self-presentation) nach \citeA{Goffman:1959} und die Theorie der Selbstoffenbarung (engl. self-disclosure) nach \citeA{Schau:2003}. Das Konzept der Selbstdarstellung beschreibt, dass in jeglicher Form der sozialen Interaktion die beteiligten Personen versuchen einen möglichst kontrollierten Eindruck von sich selber abzugeben. Üblicherweise geht dieser Vorgang mit einer gewissen Selbstoffenbarung einher, welches die bewusste oder unbewusste Offenbarung von persönlichen Informationen beinhaltet. Beide Dimensionen kombiniert, die der medienwissenschaftlichen und die der soziologischen,  führen zu einer Klassifikation von \gls{sm}, wie in folgender Tabelle dargestellt \cite{Kaplan:2010}:\par
\begin{table}[ht] \centering
	\caption{Klassifiaktion der \gls{sm}}
	\begin{tabular}{|p{3cm}|p{.5cm}|p{.5cm}|p{.5cm}|p{.5cm}|p{.5cm}|p{.5cm}|p{.5cm}|p{.5cm}|p{.5cm}|p{.5cm}|} \hline
		\rowcolor{gray} Modul & M01 & M02 & M03 & M04 & M05 & M06 & M07 & M08 & M09 & M10 \\
		\hline
		FPGA\_DATEN & & & & & X & X & & X & X & \\
		\hline
		IRQ & X & X & X & & X & & & & X & X \\
		\hline
		Nachbar Core & & & & X & & X & & X & & \\
		\hline
	\end{tabular}
	\label{tab:smKlassifiaktion}
 \end{table}	
Im Hinblick auf die soziale Präsenz und die Medienreichhaltigkeit wirken sich die kollaborativen Projekte, wie Wikipedia, und Blogs am wenigsten aus, da diese oft nur textbasiert aufgebaut sind und einen relativ simplen Datenaustausch zulassen. Auf der nächsten Ebene folgen Gemeinschaften, die Inhalte zur Verfügung stellen wie YouTube, und \gls{sns} wie Facebook, welche zusätzlich zum textbasierten Tausch auch Videos und andere Medien zum Tausch anbieten. Auf der höchsten Ebene siedeln sich virtuelle Spiel- und Sozialwelten an, die alle realen Kommunikationsfaktoren in einer virtuellen Welt versuchen abzubilden. Weiter wird anhand des Selbstdarstellungs- und Selbstoffenbarungsgrades auf der horizontalen Ebene unterschieden.\newline

%SubSec Social Network
\subsection{Social Network}\label{subsec.sn}
%SubSec Blogging
\subsection{Blogg}\label{subsec.blogg}
%UK Zukunft
\section{Zukunft}\label{sec.zukunft}
