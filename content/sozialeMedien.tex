%%%%%%%%%%%%%%%%%%%%%%%%%%%%%%%%%%%%%%%%%%%%%%%%%%%%%%%%%%%%%%%%%
%_____________ ___    _____  __      __ 
%\____    /   |   \  /  _  \/  \    /  \  Institute of Applied
%  /     /    ~    \/  /_\  \   \/\/   /  Psychology
% /     /\    Y    /    |    \        /   Zuercher Hochschule 
%/_______ \___|_  /\____|__  /\__/\  /    fuer Angewandte Wissen.
%        \/     \/         \/      \/                           
%%%%%%%%%%%%%%%%%%%%%%%%%%%%%%%%%%%%%%%%%%%%%%%%%%%%%%%%%%%%%%%%%
%
% Project     : Seminararbeit
% Title       : 
% File        : sozialeMedien.tex Rev. 00
% Date        : 10.10.2012
% Author      : Till J. Ernst
%
%%%%%%%%%%%%%%%%%%%%%%%%%%%%%%%%%%%%%%%%%%%%%%%%%%%%%%%%%%%%%%%%%

\chapter{Soziale Medien}\label{chap.sm}
\glsresetall
\textit{Ein paar reisserische Zahlen für den Einstieg wäre nicht schlecht...}\par 

In einem ersten Teil wird auf die Begrifflichkeit von \gls{sm} eingegangen.\newline
Da es sich beim Thema \gls{sm} um eine Technologie handelt, die vorwiegend in englischer Sprache verfasst und definiert ist, kann durch fehlende Übersetzung die gender-gerechte Form nicht immer berücksichtigt werden. Wo immer ein Rückschluss auf eine männliche Form vorgenommen werden kann, ist implizit auch die weibliche Form gemeint und umgekehrt.
%UK Zentrale Begriffe
\section{Begriffserklärung}\label{sec.begriff}
Gemäss \citeA{Sjurts:2011}, ist \gls{sm} (engl. social media) ein Sammelbegriff, für internet-basierte mediale Angebote, die auf einer sozialen Interaktionen basieren. Die Angebote, auch Applikationen genannt, bauen auf den Ideologien und den technischen Möglichkeiten von Web 2.0 auf \cite{Kaplan:2010} und ermöglichen das Erstellen und Austauschen von nutzererzeugten Inhalten (engl. user generated content).\newline
Web 2.0 und nutzerzeugte Inhalte bilden dabei die Grundkonzepte von \gls{sm}. Der Begriff und die Bedeutung Web 2.0 wurde 2004 neu eingeführt. Ziel war es, die Nutzung des Weltweiten Netzes (engl. World Wide Web, kurz Web oder WWW) zwischen Personen die Software entwickeln und Personen die sie nutzen, neu zu definieren (ebda.,2010). Web 2.0 stellt dabei eine Plattform dar, auf der nicht mehr einzelne Personen für den Inhalt und die Verteilung von Angebote verantwortlich waren, sondern eine Vielzahl von beteiligten Personen gemeinsam. Neben dieser Plattform, die als technologisches Fundament gilt, kann der nutzerzeugte Inhalt als die Summe aller Möglichkeiten betrachtet werden, wie die beteiligten Personen die \gls{sm} nutzen können. Dazu müssen gemäss \citeA{OECD:2007} drei grundlegende Anforderungen erfüllt sein: Erstens müssen nutzererzeugte Inhalte auf einer öffentlich publizierten Webseite zugänglich sein, oder auf einer \gls{sns} für eine ausgewählte Gruppe zur Verfügung stehen, was z.B. Mail ausschliesst. Zweitens muss der Inhalt einem gewissen Anteil an kreativen Aufwand genügen, was eine reine Kopie eines bestehenden Inhalts bereits erfüllt und drittens muss der Inhalt ausserhalb eines professionellen Umfeldes erzeugt worden sein, auf dem Hintergrund eines kommerziellen Marktes. Nutzerzeugte Inhalte wurden bereits vor dem Web 2.0, in den frühen 1980 beschrieben. Aber erst mit den Möglichkeiten von Web 2.0, technologischer Treiber (z.B. Breitband-Internetanschluss), economischer Veränderung (z.B. frei zugängliche Angebote für die Generierung von Inhalten) und soziologischer Faktoren (z.B. Entstehung einer Generation mit erheblichem technischen Fachwissen, vgl. \textquotedblleft digital natives \textquotedblright) konnten die gesamten Möglichkeiten angewendet werden und umschreiben den heutigen Begriff \gls{sm}.\par 
Eine weitere ergänzende Defintion gemäss \citeA{Ahlqvist:2008} unterteilt \gls{sm} in drei Hauptkategorien: Inhalt, Gemeinschaft und Web 2.0. Inhalt bezeichnet den von Personen generierte Inhalt, der unterschiedlicher Art sein kann (z.B. Bilder, Fotos, Videos, Kommentare, Standortinformationen, etc.). Gemeinschaft ist bereits im englischen Wort \textquotedblleft social\textquotedblright{} enthalten und bezeichnet das Bereitstellen von Inhalt auf dem Web und das Interesse am Inhalt Anderer. \gls{sm} wird dazu verwendet direkt oder mittels aufgezeichneten Daten miteinander zu kommunizieren. Die Webtechnologien und die Anwendungen, die es Personen ermöglicht über das Web zu kommunizieren, bezeichnet die dritte Kategorie Web 2.0. 
 
%UK Arten von Sozialen Medien
\section{Klassifizierung}\label{sec.unterkat}
%SubSec Social Network
\subsection{Social Network}\label{subsec.sn}
%SubSec Blogging
\subsection{Blogg}\label{subsec.blogg}
%UK Zukunft
\section{Zukunft}\label{sec.zukunft}
