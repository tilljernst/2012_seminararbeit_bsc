%%%%%%%%%%%%%%%%%%%%%%%%%%%%%%%%%%%%%%%%%%%%%%%%%%%%%%%%%%%%%%%%%
%  _____   ____  _____                                          %
% |_   _| /  __||  __ \    Institute of Computitional Physics   %
%   | |  |  /   | |__) |   Zuercher Hochschule Winterthur       %
%   | |  | (    |  ___/    (University of Applied Sciences)     %
%  _| |_ |  \__ | |        8401 Winterthur, Switzerland         %
% |_____| \____||_|                                             %
%%%%%%%%%%%%%%%%%%%%%%%%%%%%%%%%%%%%%%%%%%%%%%%%%%%%%%%%%%%%%%%%%
%
% Project     : LaTeX doc Vorlage für Windows ProTeXt mit TexMakerX
% Title       : 
% File        : diskussion.tex Rev. 00
% Date        : 7.5.12
% Author      : Remo Ritzmann
% Feedback bitte an Email: remo.ritzmann@pfunzle.ch
%
%%%%%%%%%%%%%%%%%%%%%%%%%%%%%%%%%%%%%%%%%%%%%%%%%%%%%%%%%%%%%%%%%

\chapter{Diskussion}\label{chap.diskussion} 
\glsresetall
\textbf{Kurzzusammenfassung:}\newline
Das Ziel dieser Literaturarbeit war es, einen positiven Zusammenhang zwischen der Nutzung von \gls{sm} und die Auswirkungen auf das \gls{swb} aufzuzeigen. Anhand der aufgeführten Studien kann nachgewiesen werden, dass ein gewisser Einfluss von \gls{sm} auf das \gls{swb}, durch verschiedene Faktoren hervorgeht: Eine positive Selbstdarstellung mittels Sozialer Medien wirkt sich direkt förderlich auf das \gls{swb} aus. Wobei sich eine ehrliche Selbstdarstellung indirekt, mittels Selbstoffenbarung über das soziale Umfeld oder das soziale Kapital, positiv auf das \gls{swb} auswirkt. Diese Erkenntnissen lassen darauf schliessen, dass beide Konstrukte, Selbstdarstellung und SelbstoffenbArung, eine gewichtige Rolle im Umgang mit \gls{sm} und dem \gls{swb} sind. Des Weiteren kann ein Zusammenhang zwischen der auf \gls{sm} aufgewendeten Zeit und der einhergehenden Befriedigung hergestellt werden, der sich wiederum positiv auf das \gls{swb} auswirkt. Ebenso wird der Wunsch, sich von seiner Besten Seite auf den \gls{sm} zu zeigen, zu einem erhöhten Befriedigungsempfinden führen. Die Grösse des verwendeten Netzwerks wird zudem als positiven Faktor für die Lebenszufriedenheit und soziale Unterstützung gewertet. Persönlichkeitseigenschaften wie Neurotizismus gehen negativ mit dem \gls{swb} einher, wohingegen bei der Extraversion keinen direkten Zusammenhang festgestellt werden konnte (allenfalls einen minimalen Effekt über die soziale Verbundenheit). Die Anzahl der mit einer Person vernetzten \gls{sm}-Freunde hat über die soziale Verbundenheit oder die positiven Rückmeldungen der Freunde einen weiteren positiven Einfluss auf das \gls{swb}.\par 
\textbf{Beantwortung der Fragestellung:}\newline
Im Bezug auf die eingangs gestellten Fragen konnte durch die hier aufgeführten Studien aufgezeigt werden, dass die Nutzung von \gls{sm} einen positiven Einfluss auf das \gls{swb} der Benutzer haben. Diverse Faktoren wie Selbstoffenbarung, Selbstdarstellung, aufgewendete Zeit, Anzahl befreundeter Personen, Grösse des benutzten Netzwerks und nicht zuletzt die persönlichen Eigenschaften der Benutzer hängen mit der Benutzung von \gls{sm} zusammen und haben einen Einfluss auf das \gls{swb}. Die Frage nach den Langzeitfolgen konnte in dieser Arbeit nicht beantwortet werden, da es sich bei den Studien oft um Querschnittstudien zu einem gewissen Zeitpunkt handelte und nicht um Langzeitstudien. \par 
\textbf{Interpretation der Annahmen:}\newline
Die Annahmen, dass eine gesteigerte Interaktion innerhalb der \gls{sm} zu einer Erhöhung des \gls{swb} führen könnte, kann durch den positiven Zusammenhang zwischen der Anzahl von \gls{sm}-Freunden, aufgewendeter Zeit, der Selbstdarstellung und der Netzwerkgrösse auf das \gls{swb} angenommen werden.\newline
Ob die Auswirkung auf das \gls{swb} mit der Präsenz einen direkten Zusammenhang hat, kann nur dadurch bestätigt werden, dass durch die reine aufgewendete Zeit eine positive Auswirkung auf das \gls{swb} zu verzeichnen ist. Daher kann angenommen werden, wenn die Zeit für eine andere Aktivität ausserhalb der \gls{sm} aufgewendet wird, entfällt dieser Anteil an \gls{swb}.\newline
Durch die positive Selbstdarstellung wird der Selbstwert einer Person tatsächlich erhöht, was indirekt zu mehr Vertrauen in die eigene Person führen kann. Die dritte Annahme kann somit zumindest teilweise positiv beantwortet werden.\newline
In der letzten Annahme wird die These aufgestellt, dass durch den Austausch mit Gleichgesinnten ein Gefühl der Zusammengehörigkeit entsteht, welches das \gls{swb} erhöht. Dies kann insofern bestätigt werden, dass durch die soziale Verbundenheit, die durch die Anzahl der befreundeten Personen entsteht, das \gls{swb} positiv beeinflusst wird.\newline
Hiermit kann der grösste Teil der gestellten Annahmen positiv beantwortet werden.\par  
\textbf{Methodenkritik:}\newline
Zu Selbstdarstellung, Selbstoffenbarung und Soziales Kapital: \newline
In den verwendeten Studien gemäss \citeA{Kim:2011} wurde die Messung des \gls{swb} mittels Selbsteinschätzung erstellt. Obwohl es sich dabei um eine verbreitete und adäquate Methode handelt \cite{Diener:2005}, ist sie nicht vor Fehler und Verzerrungen gefeit, die aufgrund der Antworten entstehen können (z.B.: soziale Erwünschtheit) siehe dazu \citeA{Diener:1991}. Antworten mit einer erhöhten \textit{Sozialen Erwünschtheit} könnten von Personen stammen, die sich gerne selber positiv darstellen, was zu einem positiven Zusammenhang zwischen \textit{positive self-presentation} und \gls{swb} führen könnte \cite{Diener:1991}.\newline
In der Studie von \citeA{Kim:2011} wurde vor allem Stichproben von \textit{Facebook}-Benutzern im Hochschulalter verwendet. Da sich die Nutzung auf die gesamte Öffentlichkeit erstreckt, sollte das Verhalten anhand der restlichen Benutzer untersucht werden.\newline
Die reine Nutzung von persönlichen Blogs führt nicht zwingend zu einem erhöhten \gls{swb} im realen Leben. Dies hängt stark davon ab, wie der Blog verwendet wird \cite{Jung:2012}.\par 
%SubSec Fazit
Fazit von Aufgewendeter Zeit und Befriedigung: \newline
Da der direkte Zusammenhang von \textit{self-disclosure} und \textit{satisfaction} nicht hergestellt werden konnte \cite{Special:2012}, wird die Frage aufgeworfen, was zu einer erhöhten Befriedigung \textit{(satisfaction)} führen könnte? \citeA{Sheldon:2008} stellt den Wiederwille von Personen in direkten Beziehungen zu kommunizieren, der Freude von \textit{Facebook}-Benutzern gegenüber. Sie schliesst daraus, dass extravertierte Personen einen Vorteil von \textit{Facebook} beziehen, da diese eher bereit sind zu kommunizieren. Aus diesen Annahmen lässt sich schliessen, dass die restlichen Eigenschaften der \textit{Big 5} ebenso eine Rolle spielen.\newline
Eine andere persönliche Einflussvariable, die einen Einfluss auf die Befriedigung haben könnte ist der Narzissmus \cite{Special:2012}. Es scheint plausibel zu sein, dass Personen mit einer narzisstischen Thematik sich durch \textit{Facebook} eher befriedigen lassen, da sie dadurch einen weiteren Kanal für die eigenen Bewunderung erhalten \textit{(siehe dazu Kapitel \ref{sub.traits} - \nameref{sub.traits})}.\par 

%SubSec Fazit
Fazit Benutzereigenschaften und Grösse Netzwerk:\newline
Die Auswirkungen von Extraversion und Neurotizismus auf das \gls{swb} ist sehr stark von den untersuchten Gruppen und deren Homogenität abhängig. So wurden in den Studien von \citeA{Hamburger:2000} hauptsächlich Studenten untersucht. Des weiteren wird angemerkt, dass Auswirkungen auf Grund von gender basierten Unterschieden entstehen können \cite{Special:2012}, da die Teilnahme an den Studien oft freiwillig ist und die ausgewählte Stichprobe durch den hohen Anteil an Frauen \cite{Manago:2012} nicht unbedingt einer repräsentativen Bevölkerungsgruppe entspricht. Ebenso betreffend dem Alter der \gls{sn}-Benutzer gibt es gewisse Einschränkungen, da bevölkerungswissenschaftlich der grösste Zuwachs an \gls{sn}-Benutzer Personen von 35 Jahren und aufwärts ausmachen \cite{Lenhart:2009} und davon 35\text{\%}  einen oder mehrere Konten bei einer \gls{sn} haben und eben diese Gruppe in den meisten Studien stark vernachlässigt wird \cite{Special:2012}. Alle die oben genannten Punkte legen nahe, zukünftig Studien vermehrt auf eine nicht studenten-lastige Bevölkerung auszurichten \cite{Kim:2011} \newline
Interessant ist zudem die Frage, ob es einen Zusammenhang zwischen narzisstischen Personen und der zunehmenden Anzahl von \gls{sn}-Freunden gibt \cite{Manago:2012}. Gemäss \citeA{Twenge:2008} stieg die Anzahl von Personen, die narzisstisch ausgeprägt sind, zeitgleich mit der Entwicklung von \gls{sn} an. Ob nun die Grösse der \gls{sn} der Auslöser für den Anstieg von narzisstischen Personen ist \cite{Manago:2012} oder ob sich Personen mit narzisstischen Zügen vermehrt durch die Möglichkeiten von \gls{sn} angesprochen fühlen \cite{Carpenter:2012}, da es einfacher ist sich einer breiten Öffentlichkeit zu präsentieren, ist weiter zu klären. 

\begin{itemize}
\item Weiterführende Arbeit spezifische Auseinandersetzung mit Narzissmus und SM
\item Weiterführende Studien zu Social-Capital, Selfpresentation, Self-Disclosure und Satisfaction im Allgemeinen zu SWB. Nur oberflächlichen Zugang
\item Weitere Forschung im Bereich von Virtual Role Playing Games
\end{itemize}
