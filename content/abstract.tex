%%%%%%%%%%%%%%%%%%%%%%%%%%%%%%%%%%%%%%%%%%%%%%%%%%%%%%%%%%%%%%%%%
%_____________ ___    _____  __      __ 
%\____    /   |   \  /  _  \/  \    /  \  Institute of Applied
%  /     /    ~    \/  /_\  \   \/\/   /  Psychology
% /     /\    Y    /    |    \        /   Zuercher Hochschule 
%/_______ \___|_  /\____|__  /\__/\  /    fuer Angewandte Wissen.
%        \/     \/         \/      \/                           
%%%%%%%%%%%%%%%%%%%%%%%%%%%%%%%%%%%%%%%%%%%%%%%%%%%%%%%%%%%%%%%%%
%
% Project     : Seminararbeit
% Title       : 
% File        : abstract.tex Rev. 00
% Date        : 10.10.2012
% Author      : Till J. Ernst
%
%%%%%%%%%%%%%%%%%%%%%%%%%%%%%%%%%%%%%%%%%%%%%%%%%%%%%%%%%%%%%%%%%
\thispagestyle{empty} 
\chapter*{Abstract}\label{abstract}
\textbf{Hintergrund:} Durch die rasante Entwicklung von Sozialen Medien und die zunehmende Popularität von Sozialen Netzen, werden immer mehr Leute mit dieser neuen Technologie betroffen. Diese Literaturarbeit hat das Ziel, die Auswirkungen der Sozialen Medien auf das Subjektive-Wohlbefinden anhand geeigneter Studien zu klären. \par 
\textbf{Methoden:} Für die Literaturarbeit wurden die Datenbanken Ovid, PsycInfo, PSYNDEXplus, ETH-Bibliothek und der NEBIS Datenbank nach geeigneten Studien durchsucht.\par 
\textbf{Resultate:} Sechs Hauptstudien und mehrere Nebenstudien wurden identifiziert, die sich mit den Sozialen Medien und dem Subjektiven-Wohlbefinden auseinandersetzen. Diese Forschungsergebnisse lassen vermuten, dass die Selbstdarstellung, die Selbstoffenbarung, die Grösse des verwendeten Netzwerks, persönliche Eigenschaften wie Neurotizimus und Extraversion, die aufgewendete Zeit und die Anzahl der verknüpften Freunde einen Einfluss auf das Subjektive-Wohlbefinden haben.\par 
\textbf{Schlussfolgerung:} Durch die kontinuierliche Weiterentwicklung der Sozialen Medien und die dadurch entstehenden Einflussfaktoren auf das Subjektive Wohlbefinden, werden weitere Studien folgen müssen, um vorhandene Erklärungsmodelle zu ergänzen oder sogar neu zu entwickeln.\par 

