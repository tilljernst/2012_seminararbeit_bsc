%%%%%%%%%%%%%%%%%%%%%%%%%%%%%%%%%%%%%%%%%%%%%%%%%%%%%%%%%%%%%%%%%
%_____________ ___    _____  __      __ 
%\____    /   |   \  /  _  \/  \    /  \  Institute of Applied
%  /     /    ~    \/  /_\  \   \/\/   /  Psychology
% /     /\    Y    /    |    \        /   Zuercher Hochschule 
%/_______ \___|_  /\____|__  /\__/\  /    fuer Angewandte Wissen.
%        \/     \/         \/      \/                           
%%%%%%%%%%%%%%%%%%%%%%%%%%%%%%%%%%%%%%%%%%%%%%%%%%%%%%%%%%%%%%%%%
%
% Project     : Seminararbeit
% Title       : 
% File        : abstract.tex Rev. 00
% Date        : 10.10.2012
% Author      : Till J. Ernst
%
%%%%%%%%%%%%%%%%%%%%%%%%%%%%%%%%%%%%%%%%%%%%%%%%%%%%%%%%%%%%%%%%%

\thispagestyle{empty}
\chapter*{Abstract}\label{abstract}
Die vorliegende Literaturarbeit befasst sich mit den Auswirkungen von \textit{Sozialen-Medien} auf das \textit{Subjektiven-Wohlbefinden}. Für die Literaturrecherchen wurden vorwiegend Psychologiedatenbanken wie PsycINFO, (tbd) sowie ETH nach Begriffen wie \textit{Glück}, \textit{Wohlbefinden}, \textit{social media} und \textit{happiness} gesucht. 

