%%%%%%%%%%%%%%%%%%%%%%%%%%%%%%%%%%%%%%%%%%%%%%%%%%%%%%%%%%%%%%%%%
%_____________ ___    _____  __      __ 
%\____    /   |   \  /  _  \/  \    /  \  Institute of Applied
%  /     /    ~    \/  /_\  \   \/\/   /  Psychology
% /     /\    Y    /    |    \        /   Zürcher Hochschule 
%/_______ \___|_  /\____|__  /\__/\  /    fuer Angewandte Wissen.
%        \/     \/         \/      \/                           
%%%%%%%%%%%%%%%%%%%%%%%%%%%%%%%%%%%%%%%%%%%%%%%%%%%%%%%%%%%%%%%%%
%
% Project     : Seminararbeit
% Title       : 
% File        : einleitung.tex Rev. 00
% Date        : 24.10.2012
% Author      : Till J. Ernst
%
%%%%%%%%%%%%%%%%%%%%%%%%%%%%%%%%%%%%%%%%%%%%%%%%%%%%%%%%%%%%%%%%%
\chapter{Einleitung}\label{chap.einleitung}
\glsresetall
\par
\begingroup
\leftskip=1cm
\rightskip=1.5cm
\noindent \textquotedblleft Willst Du immer weiterschweifen? Sieh, das Gute liegt so nah. Lerne nur das Glück ergreifen: Denn das Glück ist immer da.\textquotedblright, \cite{Goethe:1827}.
\par
\endgroup
% Kapitel Ausgangslage
\section{Ausgangslage}\label{ausgangslage}
Bereits in den Lehren von Aristoteles (384-322 v. Chr.) dominiert Glück als Emotion der Tugend die damalige philosophische Denkweise \cite{Mayring:2003}. Es dauerte hunderte von Jahren, bis das Glück 1998 den Weg in die Positive Psychologie fand \cite{Seligman:2003} und als übergreifender Begriff die Ziele der gesamten Positiven Psychologie beschreibt: \textquotedblleft Die gewünschten Ergebnisse der Positiven Psychologie sind Glück und Wohlbefinden\textquotedblright \ \citeyear<ebda.,>[S.409]{Seligman:2003}. \newline
\gls{sm} im Gegenzug sind, mit ihren Hauptvertretern den Sozialen Netzwerken, eine eher neuere Erscheinung und wurden mit dem ersten Sozialen Netzwerk \textquotedblleft SixDegrees.com\textquotedblright \ 1997 eingeführt \cite{Boyd:2007}. \gls{sm} bilden eine neue Art zu kommunizieren und entwickeln sich seit ihrer Einführung rasant weiter, nicht nur in Bezug auf die Popularität und die ständig steigende Benutzeranzahl \cite{Special:2012}. Jede Minute werden weltweit 10 Stunden Videomaterial auf die Plattform YouTube hochgeladen \cite{Kaplan:2010}. Facebook, als grösster Vertreter der Sozialen Netzwerke, ist bei weitem die bekannteste Seite unter den \gls{sm} \cite{Top:2012} und hat zurzeit über eine Milliarde angemeldete Benutzer \cite{Facebook:2012}. \newline
Gemäss \citeA{Bannon:2012} wurden im Juli 2012 in der amerikanischen Bevölkerung insgesamt 121.1 Milliarden Minuten für \gls{sm} aufgewendet. Gemäss einer Schweizer Studie von 2010 sind 84\% der befragten Jungendlichen bei mindestens einem Sozialen Netzwerk dabei \cite{James:2010}, wobei auch hier Facebook mit 73\% das am meisten verbreitete Soziale Netzwerk ist.\newline
Durch die zunehmende Popularität von \gls{sm} steigt auch die Kritik und die Angst vor dieser neuen Form der Kommunikation \cite{Spitzer:2012}. Täglich werden Artikel und Verhaltensanleitungen in der Tagespresse publiziert \cite{EDOB:2012}, die den Umgang mit den \gls{sm} thematisieren.

% Untertitel Fragestellung und Hypothesenb
\section{Fragestellung und Annahmen}\label{sec.fragestellung}
Die vorliegende Seminararbeit soll sich dem Einfluss von Sozialen Medien auf das subjektive Wohlbefinden annähern und insbesondere folgende Fragen beantworten: 
\begin{itemize}
\item Welche positiven Auswirkungen haben Soziale Netzwerke auf das subjektive Wohlbefinden der Benutzer?
\item Welche Faktoren im Zusammenhang mit \gls{sm} tragen zur Beeinflussung des subjektiven Wohlbefindens bei? 
\item Sind durch die Nutzung von Sozialen Netzwerken Langzeitfolgen auf das subjektive Wohlbefinden bekannt?
\end{itemize}

Zur Erörterung dieser Fragestellungen wurden die folgenden Annahmen entwickelt:

\begin{enumerate}
\item Gesteigerte soziale Interaktionen innerhalb der \gls{sm} könnten das subjektive Wohlbefinden erhöhen.
\item Die mögliche Steigerung des eigenen Status durch die Präsenz im Netz und die somit vermutete Erhöhung des subjektiven Wohlbefindens, wird nur so lange anhalten, wie man sich aktiv um seine Repräsentation im Netz kümmert.
\item Das Vertrauen in die eigene Person wird möglicherweise durch positive Selbstdarstellung erhöht.
\item Durch den Austausch mit Gleichgesinnten entsteht das Gefühl der Zusammengehörigkeit welches das subjektive Wohlbefinden positiv beeinflussen könnte.
\end{enumerate}

% Untetitel Methodisches Vorgehen
\section{Aufbau der Arbeit}\label{sec.aufbau}
In einem ersten Teil wird der Begriff des subjektiven Wohlbefindens anhand einer Begriffserklärung, dem geschichtlichen Hintergrund und der Verwendung in der Positiven Psychologie erläutert. In einem zweiten Teil werden die Sozialen Medien anhand einer Begriffserklärung und einer Klassifikation berücksichtigt. Im dritten Teil werden die beide vorangegangenen Themen und deren Konstrukte anhand aktueller Studien in einen Zusammenhang gebracht. In der abschliessenden Diskussion erfolgt eine Kurzusammenfassung der Literaturarbeit, die Ziele werden besprochen und anhand der Ergebnisse aus den Studien werden die Annahmen verglichen. Ein weiterer Teil bildet die eigentliche Diskussion, in der auf die verschiedenen Ergebnisse aus den Studien eingegangen wird. Abgerundet wird dieser letzte Teil mit einem Ausblick auf mögliche zukünftige Entwicklungen.

% Untetitel Methodisches Vorgehen
\section{Abgrenzung}\label{sec.abgrenzung}
Es werden ausschliesslich \gls{sm} im engeren Sinne der gegenseitigen Kommunikation und dem Austausch von Informationen über digitale Kanäle behandelt (Weblogs, Soziale Netzwerke, Wikipedias, Podcasts, etc.). 
Die Nutzung weiterer digitaler Medien, die für den sozialen Datenaustausch verwendet werden (wie Email, Mobile-Phone, Chatprogramme, etc.), wird in dieser Arbeit nicht behandelt.

% Untetitel Methodisches Vorgehen
\section{Methodisches Vorgehen}\label{sec.vorgehen}
Bei der vorliegenden Seminararbeit handelt es sich um eine Literaturarbeit. Zur Untersuchung der Fragestellung und der Hypothesen werden insbesondere Publikationen aus dem Bereich der Positiven Psychologie, der Glücksforschung und dem Bereich der Telekommunikation verwendet. Die Recherchen zu den einzelnen Themen wurden mittels Ovid Datenbank, PsycInfo, PSYNDEXplus, ETH-Bibliothek und der NEBIS Datenbank durchgeführt. Anhand der Schlagworte wie \textquotedblleft subjective well-being\textquotedblright \ verknüpft mit \textquotedblleft facebook\textquotedblright, \textquotedblleft social-networking\textquotedblright \ verknüpft mit \textquotedblleft authentic happiness\textquotedblright \ wurden die Datenbanken auf aktuelle Literatur durchsucht. Die Suche auf dem Gebiet der \gls{sm} wurde auf Publikationen und Studien ab dem Jahr 2005 begrenzt. 


