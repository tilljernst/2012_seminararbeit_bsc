%%%%%%%%%%%%%%%%%%%%%%%%%%%%%%%%%%%%%%%%%%%%%%%%%%%%%%%%%%%%%%%%%
%_____________ ___    _____  __      __ 
%\____    /   |   \  /  _  \/  \    /  \  Institute of Applied
%  /     /    ~    \/  /_\  \   \/\/   /  Psychology
% /     /\    Y    /    |    \        /   Zürcher Hochschule 
%/_______ \___|_  /\____|__  /\__/\  /    fuer Angewandte Wissen.
%        \/     \/         \/      \/                           
%%%%%%%%%%%%%%%%%%%%%%%%%%%%%%%%%%%%%%%%%%%%%%%%%%%%%%%%%%%%%%%%%
%
% Project     : Seminararbeit
% Title       : 
% File        : einleitung.tex Rev. 00
% Date        : 24.10.2012
% Author      : Till J. Ernst
%
%%%%%%%%%%%%%%%%%%%%%%%%%%%%%%%%%%%%%%%%%%%%%%%%%%%%%%%%%%%%%%%%%
\chapter{Einleitung}\label{chap.einleitung}
\glsresetall
\par
\begingroup
\leftskip=1cm
\rightskip=1cm
\noindent \textquotedblleft Willst Du immer weiterschweifen? Sieh, das Gute liegt so nah. Lerne nur das Glück ergreifen: Denn das Glück ist immer da.\textquotedblright, \cite{Goethe:1827}.
\par
\endgroup

% Kapitel Ausgangslage
\section{Ausgangslage}\label{ausgangslage}
Bereits in den Lehren von Aristoteles (384-322 v. Chr.) dominiert Glück als Emotion der Tugend die damalige philosophische Denkweise\cite{Mayring:2003}. Es dauerte etliche Jahre, bis das Glück 1998 den Weg in die Positive Psychologie fand \cite{Seligman:2003} und als übergreifender Begriff die Ziele der gesamten Positiven Psychologie beschreibt: \textquotedblleft Die gewünschten Ergebnisse der Positiven Psychologie sind Glück und Wohlbefinden.\textquotedblright \ \citeyear<ebda.,>[S.409]{Seligman:2003} \newline
Soziale Medien im Gegenzug sind, mit ihren Hauptvertretern den Sozialen Netzen, eine eher neuere Erscheinung und wurden mit dem ersten Sozialen Netz \textquotedblleft SixDegrees.com\textquotedblright \ 1997 eingeführt \cite{Boyd:2007, Ellison:2007}. \gls{sm} bilden eine neue Art zu kommunizieren und entwickelten sich seid ihrer Einführung rasant weiter, was die Popularität und die Anzahl Benutzer angeht \cite{Special:2012}. Jede Minute werden auf der ganzen Welt 10 Stunden Videomaterial auf die Plattform YouTube hochgeladen \cite{Kaplan:2010}. Facebook, als grösster Vertreter der Sozialen Netze, wird am meisten verwendet, ist bei weitem die bekannteste Seite unter den \gls{sm} \cite{Alexa:2011} und hat zur Zeit über eine Milliarde angemeldete Benutzer \cite{Facebook:2012}. \newline
Gemäss \citeA{Bannon:2012} wurde im July 2012 in der amerikanischen Bevölkerung insgesamt 121.1 Milliarden Minuten für \gls{sm} aufgewendet. Gemäss einer Schweizer Studie von 2010 sind 84\% der befragten Jungendlichen bei mindestens einem Sozialen Netzwerk dabei \cite{James:2010}, wobei auch hier der klare Favorit Facebook mit 73\% das am meisten verbreitete Soziale Netzwerk ist.\newline
Durch die zunehmende Popularität von \gls{sm} steigt auch die Kritik und die Angst vor dieser neuen Form der Kommunikation \cite{Spitzer:2012}. Tägliche Artikel in der Tagespresse und Verhaltensanleitungen werden publiziert \cite{EDOB:2012}, die den Umgang mit den \gls{sm} thematisieren.

% Untertitel Fragestellung und Hypothesenb
\section{Fragestellung und Annahmen}\label{sec.fragestellung}
Die vorliegende Seminararbeit soll sich dem Einfluss von \gls{SM} auf das \gls{swb} annähern und insbesondere folgende Fragen beantworten: 
\begin{itemize}
\item Welche positiven Auswirkungen haben Soziale Netzwerke auf das Subjektive Wohlbefinden der Benutzer?
\item Welche Faktoren im Zusammenhang mit \gls{sm} tragen zur Beeinflussung des Subjektiven Wohlbefindens bei? 
\item Sind durch die Nutzung von Sozialen Netzwerken Langzeitfolgen auf das Subjektive Wohlbefinden bekannt?
\end{itemize}

Zur Erörterung von dieser Fragestellung wurden die folgenden Annahmen entwickelt:

\begin{enumerate}
\item Gesteigerte soziale Interaktionen innerhalb der \gls{sm} könnten das Subjektive Wohlbefinden erhöhen.
\item Die mögliche Steigerung des eigenen Status durch die Präsenz im Netz und die somit vermutete Erhöhung des Subjektiven Wohlbefindens, wird nur so lange anhalten, wie man sich aktiv um seine Repräsentation im Netz kümmert.
\item Das Vertrauen in die eigene Person wird möglicherweise durch positive Selbstdarstellung erhöht.
\item Durch den Austausch mit Gleichgesinnten entsteht das Gefühl der Zusammengehörigkeit welches das Subjektive Wohlbefinden positiv beeinflussen könnte.
\end{enumerate}

% Untetitel Methodisches Vorgehen
\section{Methodisches Vorgehen}\label{sec.vorgehen}

Bei der vorliegenden Seminararbeit handelt es sich um eine Literaturarbeit. Zur Untersuchung der Fragestellung und der Hypothesen werden insbesondere Publikationen aus dem Bereich der Or- ganisationspsychologie berücksichtigt. Zunächst sollen im Kapitel 2 die Begriffe Kreativität und Innovation definiert und im Organisationskontext verortet werden. Eine kurze Zusammenstel- lung zu den in der Literatur identifizierten Wirkfaktoren dient als Orientierung für die weitere Arbeit. Zwei Modelle zur Kreativität werden ebenfalls im Kapitel 2 vorgestellt und zeigen den theoretischen Bezugsrahmen dieser Arbeit auf. Das Kapitel 3 widmet sich der Hypothese 1 und damit dem motivationalen Wirkfaktor, während im Kapitel 4 dem zeitlichen Wirkfaktor und da- mit der Hypothese 2 nachgegangen werden soll. Kapitel 5 untersucht die dritte Hypothese und fokussiert weitere kontextuelle Faktoren wie das Unternehmensklima und den Führungsstil. In der abschliessenden Diskussion werden die Ergebnisse zu den verschiedenen Hypothesen zu- sammengefasst, geordnet und im Hinblick auf die Fragestellung diskutiert.

